\documentclass{article}

\usepackage{geometry}
\usepackage{multicol}
\usepackage{hyperref}
\usepackage{listings}
\usepackage{float}
\usepackage{longtable}

\newfloat{listing}{th}{lst}
\floatname{listing}{Listing}

\geometry{papername=a4paper, margin = 2cm}

\title{Danlann Manual\\\small{Version 0.9.1\\For Danlann 1.2.0}}
\author{\url{wrobell@pld-linux.org}}
\begin{document}
\maketitle

\tableofcontents
\listof{listing}{Listings}

\section{Introduction}
Internet and digital cameras save a~lot of time
if one wants to publish his photos to the masses. Waiting for film
development, scanning or finding a~publisher are not necessary currently to show
own little art pieces to other people than family and friends. Just choose
photos to be published, change size and sharpness and
upload to a~hosting server on the Internet.

Gallery creation process can be automated in many ways and does not require
software with sophisticated user interface --- just specify photos and run
software to perform repeatable tasks... This way idea of Danlann was born.

\subsection{Features}
Danlann is a~command line photo web gallery generator.
It takes text gallery description files and photos as an input and produces 
nice looking web gallery of photos. Generated web gallery can be
efficiently uploaded to hosting server using FTP client with
mirror capability or rsync.

Danlann features
\begin{itemize}
\itemsep0pt
\item gallery can be edited with any text editor as format of input files
    is easy to understand
\item supports multiple input files to allow easy maintenance of gallery
    description files
\item supports albums and subalbums, a subalbum can be added to many albums
\item off--line mode (obviously), there is no need to be connected to Internet while
    editing gallery
\item easy personalization of a gallery
    \begin{itemize}
    \itemsep0pt
    \item custom CSS styles can be added to override default template
    \item custom JavaScript files can be added to add bells and whistles;
        jQuery library (\url{http://jquery.com/}) is provided and enabled by
        default to support development of gallery effects using advanced JavaScript
        programming techniques
    \item new templates can be created by reusing existing ones
    \end{itemize}
\item photo and album navigation --- previous photo, next photo, previous
    album, next album, parent album
\item customizable display of Exif data
\item photo conversion parameters can be easily and freely changed,
    Danlann supports all GraphicsMagick/ImageMagick parameters
\item gallery files consist of XHTML strict static files
\end{itemize}

\section{First Gallery}\label{example}
To create new gallery using Danlann two sets of files are needed
\begin{description}
\item[photos] photos used in gallery, they are organized in one or more
folders in a way, which is most convenient for photos author
\item[albums] text files, which describe albums, photos included in albums
    and subalbums
\end{description}
Danlann processes these two sets of files and creates a~gallery in output
directory.

Let's assume that we want to create a~gallery, which contains photos from
our holidays spent in two Irish cities --- Dublin and Galway. We would like
to title it ``Holidays in Ireland 2006''. Gallery structure can look like
\begin{multicols}{2}
\begin{itemize}
\item album \textit{Dublin}
    \begin{itemize}
    \item album \textit{Dalkey}
        \begin{itemize}
        \item photo \texttt{dsc\_0001} \textit{Dalkey Castle}
        \item photo \texttt{dsc\_0002} \textit{Dalkey Island}
        \item photo \texttt{dsc\_0003} \textit{Dalkey Port}
        \end{itemize}
    \item photo \texttt{dsc\_0004} \textit{Joyce Museum}
    \item photo \texttt{dsc\_0005} \textit{Yeats Monument}
    \item photo \texttt{dsc\_0006} \textit{Spire}
    \end{itemize}
\columnbreak
\item album \textit{Galway}
    \begin{itemize}
    \item album \textit{Musicians}
        \begin{itemize}
        \item photo \texttt{dsc\_0007} \textit{Guitar Master}
        \item photo \texttt{dsc\_0008} \textit{Beautiful Singers}
        \end{itemize}
    \item photo \texttt{dsc\_0009} \textit{Main Street}
    \item photo dsc\_0010 \textit{Bay}
    \end{itemize}
\end{itemize}
\end{multicols}

Above gallery can be defined in two album files \texttt{dublin.txt} and
\texttt{galway.txt}, see listings~\ref{example:album:dublin}
and~\ref{example:album:galway}.

\begin{listing}
\begin{lstlisting}
/dublin; Dublin
/dublin/dalkey
dsc_0004; Joyce Museum
dsc_0005; Yeats Monument
dsc_0006; Spire

/dublin/dalkey; Dalkey
dsc_0001; Dalkey Castle
dsc_0002; Dalkey Island
dsc_0003; Dalkey Port
\end{lstlisting}
\caption{First gallery example --- \texttt{dublin.txt}}\label{example:album:dublin}
\end{listing}

First album file defines \textit{Dublin} and \textit{Dalkey} albums.
\textit{Dublin} album contains three photos \textit{Joyce Museum},
\textit{Yeats Monument} and \textit{Spire}. \textit{Dublin} album
also references \textit{Dalkey} album as its subalbum.

\begin{listing}
\begin{lstlisting}
/galway; Galway
/galway/musicians
dsc_0009; Main Street
dsc_0010; Bay

/galway/musicians; Musicians
dsc_0007; Guitar Master
dsc_0008; Beautiful Singers
\end{lstlisting}
\caption{First gallery example --- \texttt{galway.txt}}\label{example:album:galway}
\end{listing}

Second album is defined in similar way. There are two albums, \textit{Galway} album references
\textit{Musicians} album as its subalbum.

\begin{listing}
\begin{lstlisting}
[danlann]
title  = Holidays in Ireland 2006

indir  = in
albums = dublin.txt galway.txt

outdir = irish-holidays-06
\end{lstlisting}
\caption{First gallery example --- configuration file}\label{example:conf}
\end{listing}

Listing~\ref{example:conf} contains configuration file for the gallery.
Danlann looks for photos in \texttt{/home/users/me/Photos} directory.
Gallery will be created in \texttt{irish-holidays-06} directory.

Run Dalnann with configuration file name as first parameter to generate
web pages, resize photos and copy required files
\begin{verbatim}
$ danlann irish-holidays.ini
\end{verbatim}
After a while it should be possible to open file
\verb$irish-holidays-06/index.xhtml$ and preview generated pages
with web browser. If there is any need for improvement, fix appropriate
files and rerun Danlann with above command.

Finally, directory \texttt{irish-holidays-06} can be sent to hosting server
and task of publishing the gallery is done. 

\section{Generated Files}\label{generated}
Danlann creates and generates different directories and files
in gallery output directory
\begin{itemize}
\item gallery index \texttt{index.xhtml} file is generated
\item directory per album is created
\item every album directory contains \texttt{index.xhtml} file
    with album's subalbum and photo index
\item set of files is generated for every photo in album directory
    \begin{itemize}
    \item photo page, it is XHTML file, i.e. \texttt{dsc\_0001.xhtml}
    \item thumbnail image, i.e. \texttt{dsc\_0001.thumb.jpg}
    \item Exif data XHTML file, i.e. \texttt{dsc\_0001.exif.xhtml}
    \item photo image (size 800x600 by default),
        i.e. \texttt{dsc\_0001.jpg}
    \end{itemize}
\end{itemize}

\section{Basic Customization}\label{customization}
Gallery can be customized to fit gallery creator needs and taste. This can
be easily achieved in many ways described in the following subsections.

% fixme: additional files: styles and javascript files

\subsection{Photo Conversion Parameters}\label{conversion}
ImageMagick (or GraphicsMagick) is used to resize photos for gallery
purposes. The size, quality and sharpening options can be overriden
in configuration file. Additional ImageMagick parameters can be specified
with \texttt{params} option. See section~\ref{conf:all} for default values
of photo processing options.

\begin{listing}
\begin{lstlisting}
[photo:thumb]
size    = 100x100>
quality = 85
unsharp =
params  = -blur 1 -border 1x1+1+1
\end{lstlisting}
\caption{Example of custom photo conversion parameters}\label{conf:photo:example:thumb}
\end{listing}

It is possible to change and extend photo conversion parameters, for example
thumbnail could be generated using following recipe
\begin{itemize}
\item size of thumbnail should be 100x100
\item use quality 85
\item do not perform sharpening operation
\item blur thumbnail 
\item add 1~pixel white border around the thumbnail
\end{itemize}
Configuration changes required to realize above recipe are presented on
listing~\ref{conf:photo:example:thumb}.

\subsection{Additional Gallery Files}\label{files}
Gallery customization often involves copying of additional files like
styles, scripts or icon images into output directory. By default, Danlann copies
\texttt{css} and \texttt{js} directories from Danlann library path. This
can be changed by setting \texttt{files} and \texttt{libpath} configuration
options, see listing~\ref{conf:copy} for examples.

\begin{listing}
\begin{lstlisting}
[danlann]
; copy standard directories (css and js) and images
; from default Danlann directory and from current directory
files   = $files images
libpath = $libpath:.
\end{lstlisting}

\begin{lstlisting}
[danlann]
; copy standard directories, images and readme file
; from default Danlann directory, /usr/local/share/galleries directory
; and from current directory
files   = $files images doc/README
libpath = $libpath:/usr/share/local/galleries/:.
\end{lstlisting}
\caption{Configuration examples used to copy additional gallery files}\label{conf:copy}
\end{listing}

\subsection{Displaying Exif Data}\label{exif}
While generating gallery, Danlann reads photo Exif data and creates
XHTML file containing photo information like aperture,
exposure time, focal length, etc. Displaying Exif data can be disabled
or changed using \texttt{exif} option (see listing~\ref{conf:exif}).

\begin{listing}
\begin{lstlisting}
[danlann]
...

# disable displaying Exif data
exif =

...
# display only Exif data like timestamp, exposure time and aperture
exif = Image timestamp, Exposure time, Aperture
\end{lstlisting}
\caption{Exif configuration example}\label{conf:exif}
\end{listing}

By default, gallery visitors can see photo parameters information like

\begin{multicols}{3}
\begin{itemize}
\item Image timestamp
\item Exposure time
\item Aperture
\item Exposure bias
\item Flash
\item Flash bias
\item Focal length
\item ISO speed
\item Exposure mode
\item Metering mode
\item White balance
\end{itemize}
\end{multicols}

Exif header names can be determined using \texttt{exiv2} command, which is
used by Danlann to get photo information.

%\subsection{CSS Styles}
%\subsection{JavaScript Effects}

\section{Input Files}\label{albums}
Danlann input file defines albums and their content like subalbums and
photos. Multiple files are accepted for easy maintenance of gallery.

Input file consists of lines. Every line can contain album,
subalbum or photo definition. A~line can be also a~comment or it can be empty.

Grammar of input file is defined on listing~\ref{grammar}.

\begin{listing}
\begin{lstlisting}
line      := album | subalbum | photo | comment
album     := slash dir '; ' title ['; ' desc]
subalbum  := slash dir
photo     := file ';' [' ' title ['; ' desc]]
title     := string
desc      := string
dir       := [A-Za-z0-9_\-/]+
file      := [A-Za-z0-9_\-]+
string    := [^;]+
slash     := '/'
comment   := ^\#.*
\end{lstlisting}
\caption{Grammar of input files}\label{grammar}
\end{listing}

Input file defines also order of subalbums and photos. 
Therefore, if order of photos presented in a~gallery needs to be changed, then
move appropriate lines in an input file.

\section{Configuration Files}\label{conf:all}
Danlann can read configuration file in \texttt{.ini} format. Danlann
recognizes several sections
\begin{description}
\item[danlann] main gallery and Danlann configuration
\item[photo:thumb] photo thumbnail configuration
\item[photo:image] photo image configuration
\item[template] template configuration
\end{description}

Table below describes all the configuration options giving references to
parts of this manual containing more valuable information.

\begin{longtable}{|l|p{3cm}|p{6cm}|p{3cm}|}
\hline
\textbf{Option} & \textbf{Default} \textbf{Value} & \textbf{Description} &
\textbf{Remarks} \\
\hline
\hline
\multicolumn{4}{|c|}{\texttt{[danlann]} section}\\
\hline
\hline
albums         &  & list of gallery input files separated by space, files determine also order of subalbums and photos & see section~\ref{example} \\
\hline
description    &  & gallery description shown on gallery index page & see also gallery title option \\
\hline
exclude        & {\small \verb+.svn|CVS|~$|\.swp$+} & files to be excluded while copying gallery additional files (it is regular expression) & see files option \\
\hline
exif           & & list of Exif headers separated by coma & see section~\ref{exif} to check default value \\
\hline
files          & \verb+css js+ & list of files separated by space to be copied from library directories to gallery output directory  &  see section~\ref{files} \\
\hline
graphicsmagick & \texttt{False} & use GraphicsMagick if \texttt{True}, ImageMagick otherwise & \\
\hline
indir          &  & list of directories separated by colon, directories contain photo files to be converted into web gallery photos & see section~\ref{example} \\
\hline
libpath        &  & list of directories separated by colon, directories contain additional gallery files  & see section~\ref{files} \\
\hline
outdir         &  & gallery output directory  & see section~\ref{example} \\
\hline
title          &  & gallery title & see section~\ref{example} \\
\hline
validate       & \texttt{False} & validate generated XHTML files if set to \texttt{True} & \\
\hline
\hline
\multicolumn{4}{|c|}{\texttt{[photo:*]} sections} \\
\hline
\hline
size & \verb+128x128>+\newline or \verb+800x600>+\newline
    & photo (thumbnail, photo image itself) size geometry as accepted by GraphicsMagick/ImageMagick
    & see section~\ref{conversion} \\
\hline
quality & \verb+90+ & output image JPEG quality ($0-100$) &  \\
\hline
unsharp & \verb-0.1x0.1+2.0+0- & image sharpening parameters as accepted by GraphicsMagick/ImageMagick &  \\
\hline
params & & additional photo conversion parameters (accepted by GraphicsMagick/ImageMagick) &  \\
\hline
\hline
\multicolumn{4}{|c|}{\texttt{[template]} section} \\
\hline
\hline
css & \verb+css/danlann.css+ & additional CSS styles to be included by generated XHTML pages & see also \texttt{files} option above\\
\hline
copyright & & copyright statement, i.e. \texttt{(cc) by john smith}, can contain XHTML tags & \\
\hline
\end{longtable}
%%
%%conf.get(self.__conf__, 'js')
%%conf.get(self.__conf__, 'onload')


\end{document}
