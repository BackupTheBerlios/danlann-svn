\documentclass[draft]{article}

\usepackage{geometry}
\usepackage{hyperref}

\geometry{papername=a4paper, margin = 2cm}

\title{Danlann Manual}
\author{\url{wrobell@pld-linux.org}}
\begin{document}
\maketitle

\tableofcontents

\section{Introduction}
Internet and digital cameras save a~lot of time
if one wants to publish his photos to the masses. Waiting for film
development, scanning or finding a~publisher are not necessary currently to show
own little art pieces to other people than family and friends. Just choose
photos to be published, change size and sharpness and
upload to a~hosting server on the Internet.

Gallery creation process can be automated in many areas and do not require
software with sophisticated user interface --- just specify photos and run
software to perform repeatable tasks... This way Danlann idea was born.

Danlann is a~command line photo web gallery generator.
It takes text gallery description files and photos as an input and produces 
nice looking web gallery of photos. Generated web gallery can be
efficiently uploaded to hosting server using rsync or FTP client with
mirror capability.

Danlann features
\begin{itemize}
\itemsep0pt
\item gallery can be edited with any text editor as format of input files
    is easy to understand
\item supports multiple input files to allow easy maintenance of gallery
    description files
\item supports albums and subalbums, a subalbum can be added to many albums
\item off--line mode (obviously), there is no need to be connected to Internet while
    editing gallery
\item easy personalization of a gallery
    \begin{itemize}
    \itemsep0pt
    \item custom CSS styles can be added to override default template
    \item custom JavaScript files can be added to add bells and whistles
    \item own templates in Python language can be developed
    \end{itemize}
\item photo and album navigation --- previous photo, next photo, previous
    album, next album, parent album
\item customizable EXIF
\item gallery files consist of XHTML strict static files
\end{itemize}

\section{First Gallery}

\section{Input Files}
- gallery file
- albums files
- configuration files

\section{Basic Customization}
- css
- js

\section{Advanced Customization}
\end{document}
